\documentclass{article}

\usepackage{graphicx}
\pagestyle{plain}
\usepackage[utf8]{inputenc}
\usepackage[T1]{fontenc}
\usepackage[unicode]{hyperref}

\author{Patryk Jędrzejko 200406}
\title{Sprawozdanie z laboratorium - PAiMSI. \\Drzewo binarne.}

\begin{document}
\maketitle

\section{Wprowadzenie}
W danym ćwiczeniu mieliśmy napisać program implementujący drzewo binarne. A dokładniej drzewo poszukiwań binarnych BST
\\\\Drzewo BST jest strukturą dynamiczną, która zawiera dane zbudowane z węzłów. Każdy węzeł może posiadać dwóch potomków oraz jednego przodka. Każdy węzeł jest również związany z kluczem.
\\\\Dla drzewa BST koszt jakiejkolwiek operacji jest proporcjonalny do wielkości drzewa, im więcej danych jest w drzewie tym operacje wykonywane są wolniej.
\\\\- Dla wersji optymistycznej koszt takiej operacji będzie wynosił O(log(n)).
\\\\- Dla wersji pesymistycznej koszt wzrasta do O(n).


\section{Wnioski:}
\begin{itemize}
  \item Implementacja drzewa binarnego jest mniej skomplikowana niż tablica asocjacyjna. Na pewno plusem drzewa jest to, iż można szybko wyszukać daną wartość za pośrednictwem klucza.
  \item Program działa poprawnie, tzn. dodaje węzły do drzewa przypisując odpowiedni klucz podany przez użytkownika, usuwa podany element, wyszukuję element o zadanym kluczy oraz wyświetla całe drzewo.
\end{itemize}

\end{document}
